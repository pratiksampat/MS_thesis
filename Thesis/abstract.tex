Precise CPU allocation is crucial to application performance and resource efficiency, but is notoriously difficult under dynamic cloud workloads. We argue that the fundamental problem is rooted in the mismatch of existing CPU allocation interface between the cloud and the OS---while the cloud represents CPU resources as a percentage quota of the host CPU (e.g., millicpu), the OS interprets CPU resources as time-shared quota slices allowed to run within a defined period. The cloud interface's disregard for periodicity stems from the fundamental difficulty of capturing fine-grained application runtime behavior in userspace. Consequently, existing solutions rely on coarse-grained, surrogate metrics such as CPU utilization, queue lengths, etc., leading to slow and imprecise allocation.

We present CATCloud, an OS extension that closes the semantic gap of cloud CPU allocation. CATCloud views CPU resources as a shared bandwidth interface and implements a millisecond-scale CPU bandwidth autotuner for quota and periodicity. Sitting in the OS scheduler, CATCloud realizes observability of fine-grained run time and yield time behavior of target applications; which was previously opaque to the userspace autoscalers. By continuously capturing historical data, it accurately estimates the short-term CPU period and quota requirements. Operating in milliseconds, CATCloud can quickly and effectively react to bursty, dynamic workloads with simple prediction algorithms. We show that CATCloud significantly outperforms state-of-the-art techniques in terms of responsiveness and precision. Our evaluation on various cloud workloads shows that CATCloud can improve CPU efficiency by up to 81\% as well as achieving performance improvements up to 152\% over existing autoscaling solutions.