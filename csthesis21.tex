% \documentclass[draftthesis,tocnosub,noragright,centerchapter,fullpagesingle,12pt]{uiuc_csthesis21}
\documentclass[tocnosub,noragright,centerchapter,fullpagesingle,12pt]{uiuc_csthesis21}

% Updated version of the ECE department's latex resources

% Use draftthesis for notes and date markings on every page.  Useful when you
%   have multiple copies floating around.
% Use offcenter for the extra .5 inch on the left side. Needed with fullpage and fancy.
% Use mixcasechap for compatibility with hyperref package, which does NOT like all caps default
% Use edeposit for the adviser/committee on the title page.
% Use tocnosub to suppress subsection and lower entries in the TOC.
% PhD candidates use "proquest" for the proquest abstract.

\makeatletter

\usepackage{setspace}  % Useful for single, 1.5, and double spacing
\usepackage[numbers, sort]{natbib}  % Useful for formatting reference section
\usepackage{url}  % Useful for URLs
%\usepackage{hyperref}  % Another package useful for URLs

\usepackage{lscape}  % Useful for wide tables or figures.
% Following command definition is from Stack Exchange: https://tex.stackexchange.com/questions/278113/single-landscape-page-with-page-number-at-the-bottom 
% It adds *rotated* page numbers to the bottom of landscaped pages to meet the Graduate College standards (see page 7 here: https://grad.illinois.edu/files/pdfs/thesis-sample-chapter-straight-numbering.pdf)
\def\fillandplacepagenumber{
	\par
	\pagestyle{empty}
	\vbox to 0pt{\vss}
	\vfill
	\vbox to 0pt{
		\baselineskip 0pt
		\hbox to \linewidth{\hss}
		\baselineskip\footskip
		\hbox to \linewidth{\hfil\thepage\hfil}\vss
	}
}

%%%%%%%%%%%%%%%%%%%%%%%%%%%%%%%%%%%%%%%%%%%%%%%%%%%%%%%%%%%%%%%%%%%%%%%%%%%%%%%
% FIGURE PACKAGES
%
\usepackage{graphicx}  % Please import figures that are *high resolution* PDFs
%\usepackage{epsfig}   % or EPS files
\usepackage{caption}
\usepackage{color}
\usepackage{xcolor}
%\usepackage{subfigure}  % Useful for subfigures
%\usepackage{subcaption}  % Useful for captioning subfigures
%%%%%%%%%%%%%%%%%%%%%%%%%%%%%%%%%%%%%%%%%%%%%%%%%%%%%%%%%%%%%%%%%%%%%%%%%%%%%%%
% TABLE PACKAGES
%
\usepackage{booktabs}  % Useful for high quality tables (e.g., you can replace \hrule with \toprule, \midrule, and \bottomrule).
%\usepackage{multicol}
%\usepackage{multirow}
%%%%%%%%%%%%%%%%%%%%%%%%%%%%%%%%%%%%%%%%%%%%%%%%%%%%%%%%%%%%%%%%%%%%%%%%%%%%%%%
% MATH PACKAGES (Comment out this section if unnecessary for your dissertation)
%
\usepackage{amsfonts}
\usepackage{amsmath}
\usepackage{amssymb}
\usepackage{amstext}
\usepackage{amsthm}


% Change numbering of definitions, lemmas, theorems, etc to meet the Graduate College standards
\theoremstyle{definition}
\newtheorem{definition}{Definition}[chapter]
\newtheorem{lemma}{Lemma}[chapter]
\newtheorem{theorem}{Theorem}[chapter]
\newtheorem{corollary}{Corollary}[chapter]
\newtheorem{conjecture}{Conjecture}[chapter]
\newtheorem{remark}{Remark}[chapter]

\renewcommand{\qedsymbol}{QED.}  % Change symbol at end of proofs to meet the Graduate College standard
%%%%%%%%%%%%%%%%%%%%%%%%%%%%%%%%%%%%%%%%%%%%%%%%%%%%%%%%%%%%%%%%%%%%%%%%%%%%%%%
% ALGORITHM AND CODE PACKAGES (Comment out this section if unnecessary for your dissertation)
%
\usepackage{listings}  % Useful for formatting code blocks, see here for further information about formatting code: https://en.wikibooks.org/wiki/LaTeX/Source_Code_Listings
\usepackage[ruled]{algorithm2e}  % Useful for formatting algorithms (pseudocode)
\numberwithin{algocf}{chapter}     % Change numbering of algorithms to meet the Graduate College standards

%%%%%%%%%%%%%%%%%%%%%%%%%%%%%%%%%%%%%%%%%%%%%%%%%%%%%%%%%%%%%%%%%%%%%%%%%%%%%%%
% COVERPAGE
%

% Uncomment the appropriate one of the following four lines:
\msthesis
% \phdthesis
%\otherdoctorate[abbrev]{Title of Degree}
% \othermasters[abbrev]{Title of Degree}

\title{CATCloud: Closing Semantic Gaps of CPU Interfaces for Precise Autoscaling in the Cloud}
\author{Pratik Rajesh Sampat}
\department{Computer Science}
\degreeyear{2024}

% Advisor name is required for
% - doctoral students for the ProQuest abstract
% - master's students who do not have a master's committee
\advisor{Professor Saugata Ghose and Professor Tianyin Xu}
% \advisor{Professor Tianyin Xu}

% Uncomment the \committee command for
% - all doctoral students
% - master's students who have a master's committee
% \committee{Professor Firstname Lastname, Chair\\
%         Professor Firstname Lastname} % etc.

\newcommand{\tianyin}[1]{\textcolor{blue}{ty:#1}}
\newcommand{\todo}[1]{\textcolor{blue}{#1}}
\newcommand{\pratik}[1]{\textcolor{violet}{ps:#1}}
\newcommand{\hq}[1]{\textcolor{red}{hq:#1}}
\newcommand{\blue}[1]{\textcolor{blue}{#1}}
\newcommand{\cmark}{\ding{51}}
\newcommand{\xmark}{\ding{55}}

\begin{document}

%%%%%%%%%%%%%%%%%%%%%%%%%%%%%%%%%%%%%%%%%%%%%%%%%%%%%%%%%%%%%%%%%%%%%%%%%%%%%%%
% COPYRIGHT
%
% \copyrightpage
% \blankpage

%%%%%%%%%%%%%%%%%%%%%%%%%%%%%%%%%%%%%%%%%%%%%%%%%%%%%%%%%%%%%%%%%%%%%%%%%%%%%%%
% TITLE
%
\maketitle

%\raggedright
\parindent 1em%

\frontmatter

%%%%%%%%%%%%%%%%%%%%%%%%%%%%%%%%%%%%%%%%%%%%%%%%%%%%%%%%%%%%%%%%%%%%%%%%%%%%%%%
% ABSTRACT
%
\begin{abstract}
Precise CPU allocation is crucial to application performance and resource efficiency, but is notoriously difficult under dynamic cloud workloads. We argue that the fundamental problem is rooted in the mismatch of existing CPU allocation interface between the cloud and the OS---while the cloud represents CPU resources as a percentage quota of the host CPU (e.g., millicpu), the OS interprets CPU resources as time-shared quota slices allowed to run within a defined period. The cloud interface's disregard for periodicity stems from the fundamental difficulty of capturing fine-grained application runtime behavior in userspace. Consequently, existing solutions rely on coarse-grained, surrogate metrics such as CPU utilization, queue lengths, etc., leading to slow and imprecise allocation.

We present CATCloud, an OS extension that closes the semantic gap of cloud CPU allocation. CATCloud views CPU resources as a shared bandwidth interface and implements a millisecond-scale CPU bandwidth autotuner for quota and periodicity. Sitting in the OS scheduler, CATCloud realizes observability of fine-grained run time and yield time behavior of target applications; which was previously opaque to the userspace autoscalers. By continuously capturing historical data, it accurately estimates the short-term CPU period and quota requirements. Operating in milliseconds, CATCloud can quickly and effectively react to bursty, dynamic workloads with simple prediction algorithms. We show that CATCloud significantly outperforms state-of-the-art techniques in terms of responsiveness and precision. Our evaluation on various cloud workloads shows that CATCloud can improve CPU efficiency by up to 81\% as well as achieving performance improvements up to 152\% over existing autoscaling solutions.  % Put the text for the abstract in a file called "abstract.tex", and it will be inserted here.
\end{abstract}

%%%%%%%%%%%%%%%%%%%%%%%%%%%%%%%%%%%%%%%%%%%%%%%%%%%%%%%%%%%%%%%%%%%%%%%%%%%%%%%
% DEDICATION (Uncomment this section if desired)
%
%\begin{dedication}
% Whatever dedication you want, for example: "To my parents, for their love and support."
%\end{dedication}

%%%%%%%%%%%%%%%%%%%%%%%%%%%%%%%%%%%%%%%%%%%%%%%%%%%%%%%%%%%%%%%%%%%%%%%%%%%%%%%
% ACKNOWLEDGMENTS
%
\begin{acknowledgments}
%\input{acknowledgements}  % Inserts contents from "acknowledgements.tex" here
\end{acknowledgments}

%%%%%%%%%%%%%%%%%%%%%%%%%%%%%%%%%%%%%%%%%%%%%%%%%%%%%%%%%%%%%%%%%%%%%%%%%%%%%%%
% TABLE OF CONTENTS
%
\tableofcontents

\mainmatter

%%%%%%%%%%%%%%%%%%%%%%%%%%%%%%%%%%%%%%%%%%%%%%%%%%%%%%%%%%%%%%%%%%%%%%%%%%%%%%%
% INSERT REAL CONTENT HERE
%

\chapter{Introduction}
\label{chp:intro}
% \section{Introduction}
% \label{sec:introduction}

Precise allocation of CPUs is crucial in the cloud setting. Cloud providers can utilize this information to efficiently provision these resources and schedule applications accordingly. As a result, users experience predictable performance and costs while running their workloads. Poor estimation of CPU limits can lead to either a loss in performance due to throttling or CPU slack, where allocated but unused CPUs collectively lower the efficiency of the data center. Estimating CPU limits upfront, however, is non-trivial especially in dynamic behavior setting where load spikes may occur in an instant. Traditionally, the user would over allocate resources to prevent performance penalities. However, this led to significantly higher costs, as well as poor utilization of the data centers resources with analysis from Google's clusters \cite{reiss_heterogeneity_nodate} observing an average CPU utilization of 60\% and Alibaba\cite {lu_imbalance_2017} observing the average CPU utilization to not exceed over 40\%.

In today's cloud landscape, users widely adopt autoscaling techniques to optimize resource utilization and cost eff1iciency. Major cloud providers like Google \cite{rzadca_autopilot_2020}, Amazon \cite{noauthor_aws_nodate}, Microsoft \cite{edb-msft_autoscale_2023}, and IBM \cite{noauthor_ibm_nodate} offer autoscaling capabilities for resource management. Additionally, the Kubernetes Vertical Pod Autoscaler (K8s VPA) \cite{noauthor_kubernetes_nodate} has emerged as a popular open-source solution in the industry. K8s VPA continuously monitors resource utilization within defined windows, providing recommendations based on statistical analyses such as 50th and 95th percentiles of past CPU usage. Cloud providers may employ various techniques, including rules-based tuning and machine learning algorithms, to accurately estimate and suggest CPU allocations for optimal performance and cost savings.

% The rise of specialized autoscaling techniques designed to target specific application behaviors \pratik{CITE FIRM, COLA, SINAN}, suggests that the problem of efficiently scaling resources remains relevant.

We observe, that there exists a fundamental discrepancy between cloud interfaces and the OS Cgroup interface. The Linux Cgroup bandwidth controller necessitates CPU limits to be specified in terms of quota and period bandwidth. However, cloud users typically request CPU limits using millicpu or vCPU, representing a percentage ratio of quota to period. Presently, leading autoscaling solutions focus solely on adjusting the quota based on observed metrics, while the period often remains constant, typically defaulting to 100 ms. This approach leads to inaccurate resource estimations, resulting in suboptimal performance with high levels of throttling, potentially leading to over-entitlement recommended by the autoscalers in the future.

New autoscaling techniques have emerged, aiming to enhance traditional threshold and utilization-based methods. These approaches leverage queuing theory, reinforcement learning, and introduce novel metrics like performance awareness. However, despite these advancements, these techniques do not account for tuning the combination of both quota and period. Along with that, these techniques also encounter issues related to reactivity and precision when making recommendations further lowering their efficacy. Due to these inefficiencies, some cloud developers requested for interfaces \cite{noauthor_fix_nodate} from orchestrators like Kubernetes to manually tune for both quota and period, while some users also called for not using quota-period mechanism altogether \cite{noauthor_avoid_nodate} \cite{noauthor_requests_2023} \cite{noauthor_for_nodate} for when dealing with dynamic workload when their periodicity is not close to default.

The core problem is that userspace autoscalers rely on surrogate metrics as a proxy to model the application behavior. The application runtime behavior while opaque to the userspace is fairly transparent within the OS kernel. A task's behavior in terms of admittance, runtime and yield can be traced within the scheduler which can be further be used to accurately model the current application behavior. We propose \textit{CATCloud}, an in-kernel CPU auto tuning solution that profiles the OS scheduler to extract the application behavior and recommend quota and period based on past behavior. CATCloud presents a lightweight approach to tracing and recommendation by minimally instrumenting the Linux bandwidth scheduler to track runqueue statistics of runtime, yield and collates them at the Cgroup-level. CATCloud can operate in the granularity of milliseconds and employs simple statistical techniques to recommend future CPU limits.

Our proposed solution addresses two primary challenges. Firstly, accurately modeling runtime behavior proves to be a complex task. Instead of relying on surrogate metrics like CPU utilization or throttle count, CATCloud extracts application runtime behavior directly from the OS scheduler. While the scheduler manages operations such as running and yielding, extracting these metrics and modeling application behavior presents difficulties. Tracing bandwidth runtime poses challenges due to the scheduler's nature of task-switching based on metrics like context switches, IO wait times, and vruntime slice expiration. These fragmented runtime and yield slices fail to accurately represent actual application runtime behavior. Initially, we outline a straw-man approach involving period-bound tracing to gather this information and highlighting its limitations using microbenchmarks. Subsequently, we introduce a novel technique called period-agnostic tracing to track fragmented runtime and yield data. This data is then aggregated across multiple cores where the application concurrently runs. The application's runtime quota is modeled as the summation of estimated runtimes, while the period encompasses the summation of runtime and worst-case yield duration.

Secondly, contemporary CPU autoscaling solutions, apart from overlooking the period parameter, often rely on heavyweight algorithms like Machine Learning and Reinforcement Learning. These algorithms demand extensive telemetry data and time to build models, resulting in slow and sluggish reactivity (several minutes) when suggesting recommendations. This delay becomes particularly undesirable when dealing with workloads exhibiting high levels of dynamic utilization. CATCloud offers a contrasting approach with its lightweight tracing infrastructure, enabling millisecond-scale tracing. This results in significantly heightened reactivity when observing spikes in application load. Moreover, CATCloud empowers users to fine-tune the level of reactivity based on their domain-specific knowledge of application behavior.

We implement CATCloud as an extension to the Linux bandwidth scheduler, providing CATCloud interfaces through the Linux CPU Cgroup. These interfaces can be leveraged by container orchestrators to enable monitoring and recommendations. To assess CATCloud's effectiveness, we deploy a custom Linux kernel on QEMU x86 KVM and utilize Kubernetes as the cloud orchestrator. Our experiment encompasses a diverse range of cloud applications, including microservices, streaming, and serverless architectures, each exhibiting varying degrees of dynamic behavior.

In our evaluation, CATCloud surpasses state-of-the-art autoscaling techniques such as Holt-Winters exponential smoothing (HW), Long Short-Term Memory (LSTM), Kubernetes Vertical Pod Autoscaler, and Autothrottle \pratik{CITE}. CATCloud demonstrates notable improvements in CPU efficiency, achieving up to XX-YY enhancement, and delivering performance gains of up to XX-YY compared to the evaluated baselines.  % Inserts content from "introduction.tex" here

\chapter{Background}
\label{chp:backgronund}
\tianyin{Need to organize the writing around the discrepancy of the interface/abstraction, instead of talking about Linux and Kubernetes.}

\section{CPU Bandwidth as the OS Abstraction}
\label{sec:os_interface}

The OS kernel exposes CPU time  through the CPU bandwidth controller.
The Linux OS kernel implements functionality to limit CPU utilization as a function of CPU bandwidth time \cite{turner_cpu_2010} wherein task groups are assigned a quota-period combination. Quota is the maximum allocated time a task group can run for within a period. 
\todo{Give clear definition.
\begin{itemize}
    \item Quota: 
    \item Period:
\end{itemize}
}
The bandwidth controller maintains a notion global quota, local quota, and bandwidth slice. \tianyin{Give clear definition:
\begin{itemize}
    \item Global quota
    \item Local quota
    \item Bandwidth slice:
\end{itemize}}
\tianyin{I have trouble to map the quota-period mental model (which I believe is a conceptual model) to this global-local-quota-slice mental model (the implementation?).}
The bandwidth controller is integrated with the existing scheduler and the control interfaces are exposed via Cgroup to the userspace \cite{noauthor_control_nodate}. We illustrate the working of the bandwidth controller using an example illustrated in Figure \ref{fig:cfs_bandwidth}. In this example, the quota is set to 15 ms and the period is set to 100 ms, and the bandwidth slice is 5ms. For each runqueue, a local quota is assigned runtime in the form of bandwidth slice. When the local quota of 5 ms is exhausted, more is then borrowed from the global quota post which the task can continue running. \tianyin{I don't really understand this sentence.} Once the global quota is exhausted at 15 ms \tianyin{so ``quota'' means ``global quota''? Local quota is just slice?}, the task group is then throttled until the next period

\begin{figure}
    \centering
    \includegraphics[width=0.5\linewidth]{paper//Figures//Background/cfs_bandwidth.png}
    \caption{Linux bandwidth controller \tianyin{no idea what each color means; don't understand the term ``borrow''}}
    \label{fig:cfs_bandwidth}
\end{figure}
\begin{figure}
    
    
\end{figure}
By default, the bandwidth slice = 5 ms, period = 100 ms, and quota is set to max which lets regular applications run unrestricted. In addition to CPU limit, the cgroup also provides statistics regarding usage, pressure, throttle, etc.

\section{Millicpu as the Cloud Abstraction}

\tianyin{Start from the abstraction first and then using Kubernetes as an example of the implementation.}
In production cloud, containers are a popular mode of deployment. Kubernetes (K8s) \cite{noauthor_production-grade_nodate} is an orchestration platform designed for automated scaling, deployment, and management of containerized applications. The fundamental building block of Kubernetes is pods. Pods are encompass the containers of an application with defined resource limits. We describe the core interfaces of CPU control and their management via autoscaling techniques

\textbf{Millicpu or Millicore.} The limits to the CPU resource is defined as a unit of millicpu \cite{noauthor_resource_nodate}. A millicpu is defined as a thousandth part of a physical CPU worth of runtime. Within the OS, millicpu translates to the ratio of quota to the period. For eg. 500 millicpu = 0.5 vCPU = 50 ms of quota, 100 ms of period. When a user tunes the millicpu limits, the period is always kept constant, likely to the default 100 ms while the quota is tuned proportionally to match the ratio.

\section{Implications}

\tianyin{Start from a high level. Is autoscaling the only implication? It's completely fine we only address autoscaling in this paper, but as a background section this needs to be broad.}

\textbf{Pod Autoscaling.} As estimating resource limits upfront proves challenging for applications, Kubernetes (K8s) offers functionality to autoscale resource limits over time based on application behavior. Specifically, for CPU resources, the Kubernetes Vertical Pod Autoscaler (VPA) \cite{noauthor_kubernetes_nodate} continuously monitors resource utilization within defined windows and offers recommendations based on statistical analyses such as the 50th and 95th percentiles of past CPU usage. However, the K8s VPA suffers from several shortcomings. Firstly, each time limits are updated, pods must be restarted, which can significantly disrupt application workflows. Secondly, the current implementation of VPA cannot be deployed alongside the horizontal pod autoscaler, a critical functionality in distributed systems on the cloud. In the following section, we explore other state-of-the-art autoscaling solutions. Although these solutions do not share the aforementioned shortcomings, they still struggle to efficiently scale CPUs due to the fundamental disconnect between OS and cloud interfaces.

\chapter{Limitations of Existing CPU autoscaling}
\label{chp:limits}
\tianyin{This section is quite unclear. What is the fundamental problem here? It should be the semantic gap, but somehow the section argues a bunch of other things, e.g., proxy metrics. Organize the fundamental limitation in the preamble followed by the resulting limitations (e.g., the need to restart and the slow response).
I think we should break this section into two: the first part should be merged to related work. The second part should be merged into 2.3.}

Currently vertical autoscalers employ various techniques \cite{lorido-botran_review_2014,qu_auto-scaling_2018} to dynamically tune CPU entitlement based on system as well as application behavior. These approaches include: (a) Rules and threshold-based \cite{edb-msft_autoscale_2023,noauthor_aws_nodate}, (b) statistical models \cite{noauthor_kubernetes_nodate,sachidananda_collective_2022,rattihalli_exploring_2019}, (c) Time series / Machine Learning / Reinforcement learning techniques \cite{qiu_firm_nodate,wang_predicting_2021,rzadca_autopilot_2020, wang_autothrottle_2023}, (d) Based on queuing theory \cite{fried_caladan_2020, ali-eldin_adaptive_2012, ousterhout_shenango_2019}, and (e) Performance aware models~\cite{bhardwaj_cilantro_2023}. 

All of these approaches above employ the use of proxy metrics (CPU utilization, throttle count, queue lengths) to estimate application behavior, with some approaches utilizing a combination of them. \tianyin{still don't really understand why it's a problem.} Approaches (a), (b), and (c) use CPU utilization as their proxy metric, wherein approaches like (b) and (c) often use heavy-weight algorithms which can lead to sluggish reactivity (several minutes) on load changes. 
\tianyin{being slow is a problem; but using proxy metrics is not.} Approaches like (d), which employ queuing lengths, are often great at identifying variations in terms of response times from their SLO; however, they often lead to imprecise recommendations in terms of whole cores or fractional cores increasing in a step function. The CPU slack caused by this is often unacceptable to cloud customers who require and pay for the precision of the fractional cores. Newer approaches that seek to eliminate the use of proxy metrics, such as (e), model based on logging the real-time performance of applications, also suffer from similar problems of reactivity and precision. \tianyin{too unstructured; hard to read. each paragraph should only make one point.}

Along with the shortcomings of reactivity and precision from the approaches described above, there is a fundamental disconnect between the CPU entitlement recommended derived from the use of proxy metrics and the OS interface \tianyin{Isn't this the first and foremost problem?! the paper should be built on top of that argument, rather than along with that.}, which requires CPU limits to be set as a function of bandwidth. As described in Section \ref{sec:Background}, cloud applications set the CPU limits in terms of millicores of vCPUs, whereas within the OS, limits are set in terms of quota and period CPU bandwidth time. All the current state-of-the-art autoscaling approaches only tune for quota while keeping the period constant (default 100 ms). We classify all the proxy metrics-based approaches mentioned above as quota-only models. Approaches (a), (b), and (c) tune the ratio of quota/period to be equal to the CPU utilization percentage recommended by the algorithm. Approaches (d) and (e) tune for quota either in terms of giving an additional CPU worth of runtime or can employ fractional scaling of quota based on the load it measures from its performance metrics.

The use of quota-only models leads to several problems. Firstly, not all ratios are created equal. For example, when an application requests for 500 millicores or 0.5 vCPU in terms of quota and period, that can theoretically translate to quota=50ms and period=100ms, or quota=5ms and period=10ms, and both are treated equally in terms of CPU limits and cost the same to the cloud customer. While overall CPU time granted to the application will be near identical, the bursts in which the runtime is granted will vary significantly. Anecdotal evidence \cite{noauthor_avoid_nodate,noauthor_runtime_nodate} suggests significant regressions to performance and cost when the application periodic behavior of bursty or long running isn't taken into account. Therefore, applications that do not conform into the default 100 ms period are significantly disadvantaged in terms of over-entitlement (slack) if the periodicity is greater, or in terms of performance if their periodicity is lower than the default, or likely both.

%In the following subsection, 
% \paragraph{Measuring quota-only models.}
We demonstrate the limitation of quota-only models using the Ebizzy micro-benchmark that simulates a web search workload~\cite{henson_ebizzy_2008}. \tianyin{need to justify this benchmark more --- I never heard of it before.}
%a regularly behaving microbenchmark and 
We tune the CPU entitlement using the Kubernetes VPA that use CPU utilization as its proxy metric (\S\ref{sec:}).
% , which employs CPU utilization as its proxy metric. While there are other proxy metric-based approaches that outperform tuning of quota for more irregular workloads, for a workload with a regular runtime pattern, the K8s VPA performs comparably to its counterparts and achieves optimal performance. 
The goal of the experiment is to show the effects not tuning period which can lead to performance and efficiency regressions, as well as the incorrect estimation of quota, which can further cause performance degradation.

% Please add the following required packages to your document preamble:
\begin{table*}[]
\resizebox{\textwidth}{!}{%
\begin{tabular}{l|lllll|l}

\textbf{Categories} & \textbf{Rules} & \textbf{Statistics} & \textbf{ML/RL} & \textbf{Queuing} & \textbf{Performance} & \textbf{CATCloud} \\
\hline
Primary Metric & CPU Util, throttle & CPU Util, throttle & CPU Util, throttle & Queue congestion & SLOs & Runtime in scheduler \\
\hline
Reactivity & \cmark & \cmark & \xmark & \cmark & \xmark & \cmark \\
\hline
\begin{tabular}[c]{@{}l@{}}Ability to Partially tune\\ CPUs w/o stepping\end{tabular} & \cmark & \cmark & \cmark & \xmark & \cmark & \cmark \\
\hline
Accuracy in quota tuning & \xmark & \xmark & \cmark & \xmark & \cmark & \cmark \\
\hline
quota-period combo & \xmark & \xmark & \xmark & \xmark & \xmark & \cmark 
\end{tabular}%
}
\caption{Vertical CPU autoscaling techniques comparison}\label{VPA-table}
\label{tab:my-table}
\end{table*}

% \subsection{Limitations of quota-only models}
% \label{sec:limitations_quota_only}
% We demonstrate the inefficiency of quota-only models using a micro-% benchmark: ebizzy \cite{henson_ebizzy_2008} that simulates a web-search like workload. 

We use an altered variant of this benchmark \cite{bhat_pratiksampatsleeping-ebizzy_2014} that introduces a sleep-wakeup patterns to display burstiness. Take an example of a workload that generally takes 15-20 ms to complete and post that is idle for 150ms. We compare against the Kubernetes VPA that employs CPU utilization as the proxy metric. CPU utilization is calculated by extracting accumulated runtime within a given period. Over a period of sliding windows K8s VPA computes 50/90/95 percentiles of CPU utilization and based on workload characteristics recommends the quota for the future inline with past CPU utilization percentage.

% \begin{figure}[h]
% \centering
% \includegraphics[scale=0.41]{paper/Figures/Sleeping Ebizzy microbenchmark - lower is better.png}
% \caption{Ebizzy micro-benchmark - 20 ms work 150 ms idle.
% \tianyin{need to show the resource figure also otherwise it looks like VPA is perfect.}
% \pratik{ack.}}
% \end{figure}

\begin{figure}[h]
    \centering
    \includegraphics[width=0.5\linewidth]{paper/Figures/Motivation/ebizzy_1.png}
    \caption{Ebizzy micro-benchmark - 20 ms work 150 ms idle - Normalized \tianyin{write the conclusion in the caption}} \label{fig:mot_ebizzy1}
    \end{figure}

% \begin{figure}[h]
%     \centering
%     \includegraphics[width=1\linewidth]{./Figures/Motivation/Ebizzy0/Sleeping Ebizzy microbenchmark CPUs - lower is better.png}
%     \caption{Ebizzy micro-benchmark CPU entitlement (millicores) - 20 ms work 150 ms idle.}
%     \end{figure}

\paragraph{The needs of adjusting periods.}
\tianyin{always start a paragraph with high-level points---what do you want to say in this paragraph?}
As shown in figure \ref{fig:mot_ebizzy1}, we first observe the latency characteristics when the application is under-provisioned to 150 millicores. In this scenario, the response time is fairly larger. This delay stems from the application's behavior: it initially runs for 15ms, but then encounters throttling until the completion of a 100ms period. Only after this interval has elapsed does the application proceed to execute the remaining requested 5ms. 
The second case is when the CPU limits are tuned based on a proxy metric like CPU utilization. We observe a sharp drop in latency, indicating the application is not heavily throttled anymore. This recommendation is akin to that suggested by the Kubernetes VPA. Even though optimal latency is achieved, since the workload behaviour is known, further optimizing for period can yield better CPU allocation. Therefore in the third case, since the idle duration of 150ms is known, the ideal periodicity of the application is runtime + idle time = 170ms. This new quota = 20ms, period = 170 ms when manually tuned within the Cgroup forms the ratio of 120 millicores, is a 40\% decrease in CPU limits when compared to the former.

The problem of quota-only models is not that they only incorrectly tune for period \tianyin{huhhh????}, but can also lead to incorrectly estimating of quota. To further demonstrate the consequences of making decisions for application runtime behavior in constant window, consider another scenario wherein a workload presents a uniform run-yield time behaviour of 40 ms runtime and 30 ms idle time. Assume the user sets the period = quota = 100 ms = 1000 millcores, which is sufficient for the current workload. From figure \ref{fig:runtime-idle-distro}, the period bound tracing algorithm functions as follows for the first three periods:
\tianyin{Is this a speculative example or is it explain figure 2? The example should be place in background}
\begin{itemize}
    \item\textbf{1st period}: 40 ms runtime, 30 ms yield time, and 30 ms runtime => Cumulative runtime = 70 ms
    \item\textbf{2nd period}: residual 10 ms runtime, 30 ms yield time, 40 ms runtime, and 20 ms yield time => Cumulative runtime = 50 ms
    \item\textbf{3rd period}: residual 10 ms yield time, 40 ms runtime, 30 ms yield time, and 20 ms runtime => Cumulative runtime = 60 ms
\end{itemize}

\begin{figure}[h]
    \centering
    \includegraphics[width=0.5\linewidth]{paper/Figures/irregular_runtime_example.png}
    \caption{Illustrative example - Runtime, idle-time distribution in 100ms period bound tracing.
    \tianyin{put a comparison with the ideal configuration}\label{fig:runtime-idle-distro}}
    \end{figure}

Assuming our goal is to minimize CPU entitlement for maximal performance, in this case the CFS bandwidth controller limits must be set to a quota of 70ms and a period of 100ms, or 700 millicores. However, this approach will result in severely unused quota in periods 2 and 3. This over-provisioning of resources leads to a direct cost to consumer for the CPU that they may not use for the majority of its running time. In contrast, if the workload behaviour was well understood; the ideal entitlement would have a quota of 40ms and a period of (40+30)=70ms, or 570 millcores instead.

Emulating the application behavior using sleeping ebizzy, we present latency characteristics in various CPU allocation scenarios (see Figure \ref{fig:mot_ebizzy2}). In the first case, we set the quota to be equivalent to the runtime of the program, i.e., 40 ms, while keeping the period constant. This results in poor performance in terms of latency, as explained in the preceding paragraph. However, steadily increasing the quota in subsequent runs leads to improved latency. As observed from the example, the lowest achievable latency is attained when the quota is set to 70 ms and the period to 100 ms, with the quota being the only variable adjusted. However, with knowledge of the application behavior, tuning the quota to 42 ms and the period to 70 ms also achieves minimal latency while imposing a lower CPU limit of 570 millicores instead of 700 millicores.


% \begin{figure}[h]
%     \centering
%     \includegraphics[scale=0.41]{paper/Figures/Sleeping Ebizzy microbenchmark - lower is better_40_70.png}
%     \caption{Ebizzy micro-benchmark - 40 ms work 30 ms idle}
%     \end{figure}

\begin{figure}[h]
    \centering
    \includegraphics[width=0.5\linewidth]{paper/Figures/Motivation/ebizzy_2.png}
    \caption{Ebizzy micro-benchmark - 40 ms work 30 ms idle Normalized \tianyin{too tired and bored to read another configuration; this figure should go with Figure 2 hand in hand}} \label{fig:mot_ebizzy2}
    \end{figure}

% \begin{figure}[h]
%     \centering
%     \includegraphics[width=1\linewidth]{paper/Figures/Motivation/Ebizzy1/Sleeping Ebizzy microbenchmark CPUs - lower is better.png}
%     \caption{Ebizzy micro-benchmark CPU entitlement (millicores) - 40 ms work 30 ms idle}
%     \end{figure}


% We summarize the limitations (Table \ref{VPA-table}) of the techniques used by the existing state-of-the-art vertical CPU autoscalers below:
% \begin{itemize}
%     \item\textbf{Use of Surrogate Metrics.} TODO
%     \item\textbf{Heavy-weight Algorithms}
%     \item\textbf{Precision.}
%     \item\textbf{Fundamental disconnect between CPU entitlement and OS interface.}
% \end{itemize}


Numerous approaches summarized in Table \ref{VPA-table} have been developed to address a combination of shortcomings related to reactivity and precision. However, these approaches encounter a fundamental challenge of estimating runtime behavior, which remains opaque in the userspace. This inability to observe, coupled with the interface mismatch between the bandwidth controller and the cloud, results in inefficiencies when recommending accurate CPU limits for applications. This motivates the need to design a solution in the layer where collecting this information is fairly transparent - the OS scheduler; which processes data regarding an application thread's runtime, yield and throttle behavior. Given that the solution operates within the kernel's scheduler, it must not impose significant computational overhead. Additionally, it should possess the capability to bridge the interface gap between cloud orchestrators and the bandwidth controller.

% In the next section describe the design of CATCloud, a light-weight in-kernel autoscaling solution to profile application behavior at the millisecond scale granularity and recommend accurate CPU limits to the userspace cloud applications on the fly.

% These approaches that exist in userspace are at a fundamental disadvantage as 

% In conclusion, current profiling tools rely on userspace proxy metrics of CPU utilization, throttle count, queue length, and other performance metrics to estimate the application behaviour due to the fact that runtime behavior is opaque to the userspace. This leads to several inefficiencies in recommending accurate CPU entitlement for applications. However, this information of how application threads are run, yielded and/or throttled is fairly transparent in the operating system. Therefore, this motivates the need for designing a light-weight in-kernel solution to profile application behavior and recommend accurate CPU entitlement on the fly.

\chapter{CATCloud Design}
\label{chp:design}
CATCloud is a vertical CPU autoscaling solution designed for applications that rely on Linux Cgroup CPU bandwidth controller to regulate utilization (e.g., containerized applications on multi-tenant cloud). Drawing from our previous discussion, where we highlighted the limitations of monitoring application behavior solely in userspace, we now bypass the necessity for userspace proxy metrics. Instead, we seamlessly integrate plug directly into the kernel scheduler to extract the operating system's view of the application, which is opaque to other autoscaling solutions. Residing in kernel space, CATCloud minimally instruments the Linux scheduler to trace and analyze application runtime behavior. It also extends the CPU Cgroup by exposing a userspace interface to finely tune the granularity of profiling by determining the length of observed history. In addition to profiling, CATCloud also exposes CPU entitlement recommendations and adjustments based on observed history and gives an option to automatically tune the CPU bandwidth based on these recommendations.

\textbf{Design Goals}: CATCloud has three main design goals:
\begin{itemize}
    \item \textit{Correctness in capturing application behavior.} CATCloud must be able to accurately capture application runtime and idle-time behavior within the scheduler and must account for events such as throttle.
    \item \textit{Lightweight profiling and recommendations.} Since CATCloud is designed as an extension to the OS scheduler, its profiling must be minimal, and the algorithm employed to analyze the application behavior must be computationally inexpensive.
    \item \textit{Usability.} Users must be able to provide hints to tune the reactivity of tracing and recommendations, along with the ability to extract and apply recommendations on the fly. These userspace hints can help reduce the overheads of tracing as well as improve the quality of the recommendations. CATCloud must implement a userspace interface that is in tandem with the already established CPU Cgroup interface
\end{itemize}

\section{Overview}

CATCloud's workflow comprises three essential components: profiling application behavior, analyzing collected runtime data, and deploying CPU entitlement recommendations. As discussed in Section 2, within a bandwidth-controlled environment, the combination of quota period and its ratio determines the CPU allocation for applications. In contrast to other autoscaling solutions, CATCloud fine-tunes both the quota and period parameters to achieve precise and efficient entitlement for the tasks within that cgroup.
% During the profiling of application behavior, the existing approach involves monitoring CPU utilization, serving as an indicator of a specific application's overall runtime, and is consequently set as the quota. However, adjusting the period variable is challenging due to a lack of cognizance regarding application's periodicity.
% \tianyin{This should be something in the background or even motivation.}


Most VPAs rely on proxy metrics, often leading to ineffective profiling of application runtime and periodicity. This results in incorrect and inefficient CPU entitlement. CATCloud addresses this challenge by leveraging the Linux scheduler to profile each runqueue scheduled by the application's cgroup. Each runqueue records the duration it runs before yielding, storing both runtime and yield time in a history buffer. Simultaneously, a global view of the cgroup runtime within a period is gathered. Once the history buffer is filled with data from both per-runqueue and global cgroup views, CATCloud computes the worst-case run/yield times and recommends quota-period based on that. These recommendations are then exposed to the userspace via the CPU Cgroup. CATCloud's lightweight implementation in the OS enables millisecond-scale accounting, ensuring high reactivity when sudden load spikes and drops are observed.

\section{Profiling}
To accurately model the current application behavior, we need techniques to observe its runtime and yield interactions within the scheduler. As described in section \ref{sec:2.1}, the behavior of the request queue (RQ) during execution and yielding is fragmented (Figure \ref{fig:challenge}), primarily due to scheduler slices. This fragmentation complicates the distinction between actual runtime and involuntary yields. Consequently, we propose two strategies. Firstly, a strawman approach involves tracing runtime within a defined period. We discuss its similarities with existing tracing methodologies, its limitations, and its applicability in specific scenarios. Secondly, we introduce a novel tracing methodology capable of capturing application behavior agnostic to the period in which the runtime resides in.
% We instrument the Linux bandwidth scheduler in order to trace the amount of runtime required by applications.

% \tianyin{Try to elaborate the goal first---what exactly you want to measure and why?}
% The general idea is to measure the amount of time an application runs for until it yields. \tianyin{Why you want to measure this? It's not all clear to me.} Quantifying this data over a period of time can then help us model the runtime behavior of applications.
% \tianyin{What's the definition of an application's ``runtime behavior''? Be precise. Try to be verbose about the description because your readers do not know as much as you.}

\begin{figure}[h]
\centering
\includegraphics[width=0.5\linewidth]{paper/Figures/tracing_challenge.png}
\caption{Challenges with runtime tracing}
\Description[Challenges with runtime tracing]{}
\label{fig:challenge}
\end{figure}

% \tianyin{Elaborate the key challenge more. What it means, with regards to the solution, when the runtime is fragmented?}
% However, runtime tracing is not as straightforward. The runtime for applications will not be granted as a contiguous quota due to various reasons such as scheduler context switches, vruntime slices, waiting for IO, etc.

% Therefore, We design two tracing methodologies with complementing characteristics for extracting application runtime behavior.
% \tianyin{It is weird to have two. As a reader, I only need to know one---the final product. You can have a strawman solution as a way to help readers understand the design challenging or as a way to build up the final solution. But it should not be two solutions and throw the ball to users to choose which one they should use.}

\subsection{Period bound tracing - strawman solution}
Similar to the CPU utilization monitoring approach described in Section 3, we design an in-kernel variant of the same. Our approach for period-bound tracing is simple: we account for cumulative runtime within a given period currently set by the cgroup controller. This approach can be viewed akin to the CPU utilization reported by container telemetry tools, with the key difference that the data is collected in-kernel and accounted for every single period. 

\begin{algorithm}
\caption{Period Bound Tracing}\label{alg:pbt}
\begin{algorithmic}
\Require Task bound by the CFS Bandwidth controller
\Require $quota != max$
\Require $recommend.status == true$
\State $tot\_runtime \gets 0$
\State $yeildtime \gets 0$
% \State $N \gets n$
\While{$curr\_period \leq default\_period$}
\If{$target\_runtime$ is assigned}
    \State $tot\_runtime \gets tot\_runtime + runtime\_slice $
\EndIf
\EndWhile
\State $yeildtime \gets period - tot\_runtime$
\end{algorithmic}
\end{algorithm}

As described in Algorithm \ref{alg:pbt}, in this approach, only the runtime duration is accounted for, while the yield (idle) duration is implicit at the period boundary. For example, in a 100ms period, if the cumulative runtime slices accounted for is 30ms, then the yield duration is 100 - 30 = 70ms. This inability to consider the period, does not report the true utilization of an application and hence tunes incorrectly for the quota as well. While this approach incurs the same drawbacks discussed in section \ref{sec:limitations_quota_only}, it does present utility in some cases when applications experience throttle described in section \ref{sec:throttle_tracing}

% The implications of using this algorithm are that only the quota can be tuned for a predetermined constant period. In addition, this inability to consider the period, does not report the true utilization of an application and hence is inadequate in modelling the application behavior as described in Section 3.1. 

% \hq{I feel like it's not because of this algorithm that only the quota can be tuned, i.e., this is not the implication. It's that this monitoring approach does not report the "true" utilization, i.e., it misses the case that the process may want to run but it's simply throttled within each period.}

% \tianyin{I think this solution is just a strawman so I think it can be placed here with the goal of building the next, real one.}

% \tianyin{The remaining should go to a motivation section}
% \textbf{Shortcoming of the approach.} Aggregating runtime within a single period although simple and fast can be an unreliable approach and can lead to an uneven distribution of traced runtime-yield time information, as runtime can span over periods. Moreover, since a fixed period is always used, an application may not fit into the default periodicity which can further lead to a seemingly erratic distributions of utilization even in a stable running workload.

% Consider an hypothetical scenario wherein a workload presents a uniform run-yield time behaviour of 40 ms runtime and 30 ms idle time. Assume the user sets the period = quota = 100 ms = 1000 millicores, which is sufficient for the current workload. The period bound tracing algorithm functions as follows for the first three periods:
% \begin{itemize}
%     \item\textbf{1st period}: 40 ms runtime, 30 ms yield time, and 30 ms runtime => Cumulative runtime = 70 ms
%     \item\textbf{2nd period}: residual 10 ms runtime, 30 ms yield time, 40 ms runtime, and 20 ms yield time => Cumulative runtime = 50 ms
%     \item\textbf{3rd period}: residual 10 ms yield time, 40 ms runtime, 30 ms yield time, and 20 ms runtime => Cumulative runtime = 60 ms
% \end{itemize}

% Assuming our goal is to minimize CPU entitlement for maximal performance, in this case the CFS bandwidth controller limits must be set to a quota of 70ms and a period of 100ms, or 700 millicores. However, this approach will result in severely unused quota in periods 2 and 3. This over-provisioning of resources leads to a direct cost to consumer for the CPU that they may not use for the majority of its running time. In contrast, if the workload behaviour was well understood; the ideal entitlement would have a quota of 40ms and a period of (40+30)=70ms, or 570 millcore. This is also a fundamental shortcoming for the state-of-the-art container telemetry.


% In summary, relying solely on period-bound tracing is inadequate for modeling application runtime behavior and can result in over or under allocation of CPU resources. Therefore to compliment this we have designed a period agnostic tracing model as well.

\subsection{Period agnostic tracing}
While our evaluation of period-bound tracing (Section \ref{sec:limitations_quota_only}) demonstrates a lightweight and simple approach, it presents significant shortcomings when the periodicity of the applications isn't in line with the default period. Therefore, in this section, we design a period-agnostic tracing model to profile application runtime and yield time behavior whose periodicity can either be lower or greater than the currently defined period boundaries.

\begin{algorithm}
\caption{Period Agnostic Tracing}\label{alg:pat}
\begin{algorithmic}
\Require Task bound by the CFS Bandwidth controller
\Require $quota \ne max$
\Require $recommend.status == true$
\State $accumilated\_runtime \gets 0$
\State $curr\_yieldtime \gets 0$
\State $prev\_vruntime\gets 0$
\While{$\_\_assign\_cfs\_rq\_runtime()$}
\If{$yieldtime \ne 0$}
    \State $curr\_yieldtime \gets rq\_clock() - yieldtime$
\EndIf
\State $corr\_yieldtime = curr\_yieldtime - prev\_vruntime$
\Comment{clock keeps ticking from the moment it is queued}
\If{$accumilated\_runtime \ne 0$ and $corr\_yieldtime \ge cfs\_bandwidth\_slice$}
    \State $ yeildtime\_hist[idx] \gets corr\_yieldtime$
    \State $ runtime\_hist[idx] \gets rq\_clock() - accumilated\_runtime - corr\_yieldtime$
    \State $ idx \gets idx + 1$
    \State $ accumilated\_runtime \gets 0$
\EndIf
\State $curr\_yieldtime \gets rq\_clock() $
\State $prev\_vruntime\gets vruntime $
\If{$accumilated\_runtime == 0$}
    \State $accumilated\_runtime \gets rq\_clock() $
\EndIf
\EndWhile
\end{algorithmic}
\end{algorithm}

The core idea described in algorithm \ref{alg:pat}, is that we want to account for runtime per runqueue until it self-yields. However, the runqueue can yield for several reasons, with the most common reason being the expiry of the CFS bandwidth slice. Therefore, the primary approach we employ is to account for runtime when the yield duration is greater than the scheduler bandwidth slice.

To achieve this, two runqueue clocks are maintained: one for runtime and another for yield time. The yield time clock is primed each time a runqueue is assigned a vruntime. The runtime clock is primed either if it wasn't previously primed before or when a legitimate self-yield is identified. In a low contention scenario, naturally, when a runqueue has expired its CFS bandwidth slice but needs to run and has its quota remaining, it will be assigned runtime to be scheduled in the next bandwidth slice. Therefore, to identify a self-yield, we observe when the runqueue is scheduled and check if the yield time clock is greater than the scheduler bandwidth slice. This, along with the check of the absence of throttle in the past periods and subtracting the current yield duration, signifies the current periodic runtime cycle.

\subsection{Tracing during throttle}
\label{sec:throttle_tracing}
When a task requires more runtime than the user-set quota within a period, it will be involuntarily yielded due to throttle. This means that the task is evicted from the queue for the rest of the period and is re-queued only in the next period when the quota replenishes.

We characterize throttle behavior into three main categories: light, medium, and high degrees of throttle, and describe strategies for tracing during these conditions.

During light throttle, the period-agnostic algorithm will experience higher runtime and yield time durations, potentially leading to incorrect recommendations. To accurately model the runtime behavior, we must take into account the amount of time the task was throttled and make corrections based on that. As described in Algorithm 3, we maintain a throttle clock to record the amount of time a task stayed throttled. When the task is reset in the queue to be unthrottled, we adjust the yield time and runtime clocks forward by the duration the runqueue was throttled for.

\begin{algorithm}
\caption{Throttle correction}\label{corr}
\begin{algorithmic}
\Require $quota \ne max$
\Require $recommend.status == true$
\Procedure{throttle\_cfs\_rq}{cfs\_rq}
  \State $cfs\_rq.throttled\_clock \gets rq\_clock(rq)$
\EndProcedure

\Procedure{unthrottle\_cfs\_rq}{cfs\_rq}
  \State $curr\_throttle\_time \gets rq\_clock(rq) - cfs\_rq.throttled\_clock$
  \If{$yieldtime \ne 0$ and $accumulated\_runtime \ne 0$}
       \State $yieldtime \gets curr\_throttle\_time + yieldtime$
       \State $accumulated\_runtime \gets curr\_throttle\_time + accumulated\_runtime$
  \EndIf
\EndProcedure
\end{algorithmic}
\end{algorithm}

Experiencing some amounts of throttle is inevitable in the life-cycle of an auto-tuned application and algorithm \ref{corr} successfully mitigates the effects of capturing the runtime duration. In the case an application experiences medium amounts of throttle, momentarily changing the application behavior; then period bound tracing provides a useful recommendation baseline. As deliberated in the past sections, often period bound tracing may result in higher, incorrect entitlements especially in the case of altered behavior. However, this can significant aid in quickly reducing throttle and stabilizing the workload for a more accurate runtime profiling.

Lastly, if an application experiences high degrees of throttle, then the application behavior itself can be severely affected, and correction factors may prove to be inadequate. Therefore, as described in Algorithm \ref{quota_scaling}, when throttle constitutes the majority ($\ge 50\%$) of the period-bound history size, then in that case, for a short duration of time (five bandwidth periods), the quota is scaled to over-provision for the application to stabilize the tracing. The amount of overprovision is determined by an exponential scaling factor of 2 CPUs. Once a valid runtime is discovered, the over-entitlement-based tracing is disabled.
\begin{algorithm}
\caption{Quota Scaling}\label{quota_scaling}
\begin{algorithmic}
\Require $quota \ne max$
\Require $recommend.status == true$
\State $ulim\_interval \gets 5$
\State $curr\_throttle \gets 0$
\State $cpu\_scale \gets 1$

\Procedure{do\_sched\_cfs\_period\_timer}{}
  \State $throttled = !list\_empty(throttled\_cfs\_rq)$
  \If{$curr\_throttle \ge period\_bound\_history$}
    \If{ $ulim\_interval \ge 0$}
        \State $quota \gets quota + cpu\_scale$
        \State $ulim\_interval \gets ulim\_interval - 1$
        \State $cpu\_scale \gets cpu\_scale * 2$
     \Else
        \State $ulim\_interval \gets 5$
        \State $cpu\_scale \gets 1$
    \EndIf
  \EndIf

\EndProcedure
\end{algorithmic}
\end{algorithm}

\section{CPU Period and Quota Recommendation}
% \hq{Change the title to "CPU Period and Quota Recommendation"?}\\
% \tianyin{I assume that decision making is to figure out the best period and quota. It's not?}
% \pratik{It is, updated the text to make that clear}
Runtime information for both Period-Based and Period-Agnostic tracing is stored in history buffers. When the user-tunable history buffers are full, the decision-making algorithm considers this path behavior to determine the best period and quota limits for the future.

To determine the quota-period combination, the 99th percentile of runtime and yield time duration are calculated from both period-based and agnostic history buffers. Quota is attributed to the runtime, while the period is determined by the summation of runtime and yield time. The ratio of quota to period is compared for both period-bound and period-agnostic techniques, and the lower ratio, indicating lower CPU limits, is then recommended to be applied.

If the \textbf{recommended.status} CPU Cgroup interface (described in Section \ref{sec:multicore}) is set to \textit{recommend-only} mode, only the \textbf{recommend.max} is updated for the user to manually make a decision based on these insights. If the status is set to \textit{auto mode}, the period and quota of the task are updated as well, and tracing for the following periods is based on this new quota and period.

\section{Multicore Support}
\label{sec:multicore}
Tasks can often spawn multiple threads simultaneously on multiple cores. Therefore, for period-agnostic tracing, CATCloud profiles all the RQs separately and models the quota and period for each. During each assign\_cfs\_rq, it goes through the entire list of active RQs and computes cumulative runtime and period, which forms the period-agnostic recommendation. However, as described in the past sections RQs can yield briefly for several reasons from scheduler ticks to throttle. This can lead to some genuine RQs that are active not being present in the list during the collation of all the RQs, resulting in incorrect CPU limit recommendations. To mitigate this behavior, RQs are added to the active list immediately when they appear and are persisted for the entirety of a single period, after which the contents of the active list are purged. This ensures that all active RQ behaviors are considered when a decision is made. One downside of this approach is that RQs that have been dequeued and will not enqueue anytime soon can influence the recommendation and cause over-entitlement. However, this case is less frequent, and even in the cases of over-entitlement, the effect will be brief due to the dequeued RQ being removed from consideration within the span of a single bandwidth period.

\section{Implementation}
We enhance the v6.3 Linux scheduler by introducing per-bandwidth-controller, per-RQ tracking of runtime and yield. Within the \texttt{struct cfs\_rq}, we implement heuristics to monitor behavior independent of time periods. Each RQ includes raw current clock values, historical data, and the 99th percentile values of both runtime and yield.  Additionally, we extend the \texttt{struct cfs\_bandwidth} to incorporate similar statistics for period-bound tracing mechanisms. Furthermore, we integrate variables for externally controlled interfaces of status, history, and max values within the same structure.



% \subsection{Interface}

\textbf{Interface.} We augment the Linux CPU Cgroup interface to add user-space knobs to control CATCloud.
\dirtree{%
.1 /sys/fs/cgroup/cpu.
.2 cpu.recommend.status.
.2 cpu.recommend.history.
.2 cpu.recommend.max.
}

The extended controls and their definitions are as follows:
\begin{itemize}
    \item\textbf{cpu.recommend.status}: Tribool value <0/1/2>
        \begin{itemize}
            \item $0 \to off$: Disable CATCloud
            \item $1 \to recommend\_only\_mode$: Enable tracing, export resource recommendations to cpu.recommend.max
            \item $2 \to auto\_mode$: Enable tracing, and automatically apply period and quota recommendations to cpu.max
        \end{itemize}

    \item\textbf{cpu.recommend.history}:
        \begin{itemize}
            \item <period bound history size , period agnostic history size>
        \end{itemize}

    Size of the history buffer of runtimes for both period bound and period agnostic tracing.

    The core idea behind the history sizes is to control the aggressiveness of suggestions for different kind of applications. It can so be that if the history is set too small then accurate runtime behavior may not be captured and if it is too large then stale past may taint the buffer.
    % The core idea is to provide control over the algorithm aggressiveness as if the history is too small then accurate runtime behavior may not be captured and if it is too large then stale past may taint in capturing current runtime behavior.
    \item\textbf{cpu.recommend.max}:
        \begin{itemize}
            \item <recommended quota>, <recommended period>
        \end{itemize}
    Read only file that mimics the cpu.max file format, that presents the current quota and period recommended by CATCloud.

    The cpu.recommend.max file is updated both in the recommend\_only mode as well as the auto mode for the interface cpu.recommend.status with the difference that the recommendations are only suggested for the former while automatically applied for the latter. In the latter case cpu.max and cpu.recommend.max will display the same entitlement.
\end{itemize}

\chapter{Evaluation}
\label{chp:evaluation}
\hq{
Our experiments addressed the following research questions:
\begin{itemize}
    \item Does CATCloud provide the same level of performance (regarding latency and throughput) with less CPU allocation? To understand if CATCloud can save resources by better handling throttling according to application behaviors.
    \item Does CATCloud achieve better performance (lower latency or higher throughput) when CPU allocations are the same? To understand how much benefit each application enjoys from less throttling due to misconfigured CPU periods.
    \item Can CATCloud handle bursty workloads (workload variation and load spikes)? There can be latency spikes but how fast CATCloud is able to autoscale compared to baselines?
    \item How does CATCloud perform in multicore settings?
    \item How does CATCloud perform in multi-application settings? (not sure if applicable)
    \item How does the CATCloud tracing component perform in light, medium, and high degrees of throttling?
    \item What is the overhead of CATCloud?
\end{itemize}
}

\hq{
Plan\\
Exp (1): For a given workload (e.g., fixed application + fixed arrival rate), set an SLO on latency or throughput, and compare if CATCloud requested less CPU allocation to meet SLO than the baselines.
\\
Exp (2): For a given workload, set the CPU allocations the same, and compare if CATCloud can better configure periods and quota according to application behaviors to achieve higher throughput or lower latency than the baselines.
\\
Exp (3): For a workload spike, see if CATCloud and baselines can autoscale to mitigate the performance degradation and avoid overprovisioning, compare (1) how much is the degradation, (2) how fast each method mitigates the degradation,
and (3) how much CPU allocation is used during the autoscaling process.
Then after the spike, see if CATCloud and baselines can scale down, and if so, what's the duration and the resource requested.
\\
Exp (4): Multicore
\\
Exp (5): Multi-application sharing cores
\\
Exp (7): Overhead of CATCloud (should be related to how frequently CATCloud does the profiling and makes the recommendation for period/quota)
\\
Exp (8): Microbenchmark to understand how much benefit CATCloud can lead to when the Ebizzy traffic is scaled from 0 to 100\%. CPU or I/O workloads.
\\
Exp (9): Ablation study on several parameters used in CATCloud, e.g., (1) the 99th percentile of runtime and yield time duration is used for recommendation, (2) the length of the history buffer, (3) what if history buffer is empty, (4) parameters in Algo. 4 ..., (5) CATCloud tracing component in light, medium, and high degrees of throttling
}

We assess CATCloud's capacity to achieve peak performance with reduced resource utilization compared to state-of-the-art heuristic and ML-based autoscalers. Across various applications, CATCloud consistently maintains the lowest latency while utilizing up to 56\% fewer CPU resources for Sleeping-ebizzy, 2.6\% less CPU entitlement for web-search, and 25.33\% lower CPU usage for media-streaming in Cloudsuite. In the case of the HotelReservation benchmark under simple load conditions, CATCloud outperforms existing solutions by up to 10\% in terms of performance and requires up to 8\% fewer CPU resources. Moreover, under dynamic load conditions, CATCloud demonstrates a performance advantage of up to 56\% compared to its contemporaries.

\section{Methodology}

\textbf{Benchmark applications.} We deploy three cloud applications: \textit{(1)} Sleeping Ebizzy microbenchmark \cite{bhat_pratiksampatsleeping-ebizzy_2014} to simulate webpage traffic. \textit{(2)} EPFL CloudSuite \cite{ferdman_clearing_2012} - web search and media streaming benchmarks, and Hotel-Reservation from DeathStarBench \cite{gan_open-source_2019}. These applications are representative of real-world web-based cloud applications. These display varied application behaviors from streaming to microservices that demonstrate stateless and data services.

\textbf{Comparison.} CATCloud is compared to \textit{(1)} Kubernetes CPU Vertical Pod Autoscaler (K8s VPA) \cite{noauthor_kubernetes_nodate}, \textit{(2)} Holt-Winters exponential smoothing (HW), and \textit{(3)} Long Short-Term Memory (LSTM) autoscaler. \cite{wang_predicting_2021}.

K8s VPA periodically (15 seconds) monitors the CPU utilization and recommends scale-up or scale-down millicore limits for the pod that it is attached to (every 300 seconds). The autoscaler also maintains a history of past runs and recommends initial limits based on its past run behavior as well.

We implement the the HW and LSTM strategies presented in \cite{wang_predicting_2021} and integrate it into a userspace autoscaler for real-time prediction. The implementation uses the weights supplied as-is and makes a recommendation approximately every 3 minutes.

% To evaluate the best-case performance, we compute the 99 percentile CPU utilization using docker stats (Docker P99) for the course of an applications entire run. This is a retrospective approach, wherein the workload for its first deployment runs unrestricted. During the unrestricted run, CPU utilization is collected at the granularity of 100ms. The subsequent run with the same runtime parameters then bear the 99P utilization as its CPU entitlement.


\textbf{Experiment setup.} We deploy CATCloud on a patched Linux 6.3, on a testbed of x86 KVM QEMU - 32 Cores, 32 GB. The benchmarks are containerized unrestricted in all resources except CPU with are autoscaled. Deployments are performed on the original container images and is managed by Docker and Kubernetes. K8s VPA is set up as an add-on to the pod deployment. HW and LSTM are tested using a custom userspace autoscaler that resides in the host, monitors the telemetry, and applies the recommendation directly on the Cgroup interface. In the case of CATCloud, we identify the container's cgroup that requires to be auto-tuned and activate tracing and recommendation using the cpu.recommend.status knob.

% For docker P99 -  CPU utilization is captured via a bash script in user-space for the entirety of the run and post the run 99 percentile CPU utilization is computed and new limits are set for the next run. 

\section{Cloud Benchmarks}

\textbf{sleeping ebizzy microbenchmark.} Ebizzy \cite{henson_ebizzy_2008} is a web traffic simulation benchmark. We used a custom variant \cite{bhat_pratiksampatsleeping-ebizzy_2014} to introduce burstiness. Starting with the experiment described in Section 
\ref{sec:limitations_quota_only}, we evaluate the two baselines against CATCloud (Figure \ref{ebizzy_latency} for latency, Figure \ref{ebizzy_ent} CPU limits) to ascertain if our approach is able to attain the hypothetical CPU limits minima while sustaining maximum performance (lowest latency).
K8s VPA recommends and autotunes the median CPU limits to 200 millicores, while also achieving close to the ideal latency. HW and LSTM perform poorly and require 164 and 175 millicores respectively. Lastly CATCloud, achieves ideal latency, while median CPU limits recommended is quota=22 ms, period=171ms (128 millicore). CATCloud offers the same performance characteristics of k8s for a further 56.25\% lower CPU limits. In terms of HW and LSTM, CATCloud performs 150\% better in terms of latency with 22\% and 26.8\% CPU allocation improvements.
\begin{figure}[h]
  \centering
  \includegraphics[width=0.5\linewidth]{paper/Figures/Evaluation/HW_LSTM_Data/ebizzy_latency.png}
  \caption{Sleeping Ebizzy - latency (lower is better)}\label{ebizzy_latency}
  \Description[Sleeping Ebizzy - latency]{}
  \end{figure}

  \begin{figure}[h]
    \centering
    \includegraphics[width=0.5\linewidth]{paper/Figures/Evaluation/HW_LSTM_Data/ebizzy_CPU.png}
    \caption{Sleeping Ebizzy - CPU entitlement millicores (lower is better)}\label{ebizzy_ent}
    \Description[Sleeping Ebizzy - CPU entitlement (millicores]{}
    \end{figure}

  % \begin{figure}[h]
  %   \centering
  %   \includegraphics[scale=0.21]{paper/Figures/Evaluation/Sleeping_ebizzy_combined.png}
  %   \caption{Sleeping Ebizzy benchmrk latency and CPU)}
  %   \Description[Sleeping Ebizzy - CPU entitlement (millicores]{}
  %   \end{figure}


\textbf{EPFL Cloudsuite.} A modern benchmarking suite to evaluate popular cloud services such as data-serving, web-search, media streaming, etc. \cite{ferdman_clearing_2012}. Within Cloudsuite we evaluate two compute heavy benchmarks - web search and media streaming for varying degrees of resource utilization scale. The results are normalized for ease of analysis. The metrics of measurement here are throughput and CPU entitlement.

In the case of a stable running web-search benchmark (Figure \ref{epfl_web-search} - K8s VPA, HW and LSTM, all performs close to peak performance. K8s VPA requires 20\%, LW 8.6\%, and LSTM 2.6\% higher CPUs compared to CATCloud.

In the case of the media streaming benchmark (Figure \ref{epfl_media}) - K8s VPA performs 152.5\% worse in terms of performance, with HW and LSTM performing at par to peak. In terms of efficiency, K8s, requires requires 25.33\%, HW requires 62.09\%, and LSTM 81.25\% higher higher CPU entitlement compared to CATCloud.

\begin{figure}[h]
  \centering
  \includegraphics[width=0.5\linewidth]{paper/Figures/Evaluation/HW_LSTM_Data/EPFL_websearch.png}
  \caption{EPFL Cloudsuite - Web search}\label{epfl_web-search}
  \Description[EPFL Cloudsuite - Web search]{}
  \end{figure}

  % \begin{figure}[h]
  %   \centering
  %   \includegraphics[width=1\linewidth]{paper/Figures/Evaluation/HW_LSTM_Data/EPFL_media_streaming.png}
  %   \caption{EPFL Cloudsuite - Web search CPU entitlement}\label{epfl_media}
  %   \Description[EPFL Cloudsuite - Web search CPU entitlement]{}
  %   \end{figure}

    \begin{figure}[h]
      \centering
      \includegraphics[width=0.5\linewidth]{paper/Figures/Evaluation/HW_LSTM_Data/EPFL_media_streaming.png}
      \caption{EPFL Cloudsuite - Media Streaming} \label{epfl_media}
      \Description[EPFL Cloudsuite - Media Streaming]{}
      \end{figure}

\textbf{Hotel Reservation - DeathStarBench.} The hotelReservation benchmark as part of the DeathStarbench \cite{gan_open-source_2019} suite simulates a microservice application. The workload contains many services that can be either independently or collectively controlled. We choose to control the CPU entitlement recommendation individually based on each service (recommend, search, reserve, etc). This allows finer garrulity in monitoring (for all the baselines), and avoids tuning and tainting of services that either display low or high levels of utilization. The results are normalized for ease of analysis. The metrics of measurement here are P99 Latency and CPU entitlement.


If the results are viewed from the best case recommendations from all the autoscalers (Figure \ref{deathstar_overall}) on a stable load, CATCloud performs better than K8s (10\%), HW (11.95\%) and LSTM(10.5\%) in terms of latency. In terms of efficiency, K8s, requires requires 8\%, HW requires 12.4\%, and LSTM 8.9\% higher higher CPU entitlement compared to CATCloud.

\begin{figure}[h]
  \centering
  \includegraphics[width=0.5\linewidth]{paper/Figures/Evaluation/HW_LSTM_Data/deathstar_overall.png}
  \caption{DeathStarBench - Hotel Reservation}\label{deathstar_overall}
  \Description[DeathStarBench - Hotel Reservation]{}
  \end{figure}

When dealing with long-running workloads characterized by fluctuations, the limitations of algorithmically dense approaches become evident. Analysis of figures \ref{deathstar_perf} and \ref{deathstar_cpu} reveals significant performance drops occurring between the 6-12 minute marks in response to load changes. During this period, K8s VPA, HW, and LSTM consistently either overprovision or underprovision CPU resources. As a result, performance regresses by up to 56\% compared to CATCloud, which demonstrates remarkable consistency by maintaining performance close to peak levels due to its high reactivity.

\begin{figure}[h]
  \centering
  \includegraphics[width=0.5\linewidth]{paper/Figures/Evaluation/HW_LSTM_Data/deathstar_series_performance.png}
  \caption{DeathStarBench - Hotel Reservation Performance higher is better}\label{deathstar_perf}
  \Description[DeathStarBench - Hotel Reservation]{}
  \end{figure}

\begin{figure}[h]
  \centering
  \includegraphics[width=0.5\linewidth]{paper/Figures/Evaluation/HW_LSTM_Data/deathstar_series_cpu.png}
  \caption{DeathStarBench - Hotel Reservation CPU lower is better}\label{deathstar_cpu}
  \Description[DeathStarBench - Hotel Reservation CPU lower is better]{}
  \end{figure}


% \begin{itemize}
%   \item \textbf{Microbenchmark:} To motivate on the case where it is impossible for the current solutions to ever deduce this entitlement decision. Ebizzy seems like a good candidate for this controlled experiment

%   \item \textbf{EPFL CloudSuite:} To show the cloud workload usecase - web search and media streaming. Both of which have erratic CPU utilizations

%   \item \textbf{Deathstarbench:} To show the microservices usecase. Hotel Reservation can be one of those benchmarks

%   \item \textbf{Serverless:} Pick one or two serverless workloads
% \end{itemize}

\chapter{Related Work}
\label{chp:related}
We now review related work on resource scaling for workloads in cloud setting, with a specific focus on CPU allocation management, and the techniques employed to model application behavior to predict future requirements.

Scaling of cloud applications can either be horizontal or vertical. Horizontal scaling is employed when an application's resource requirements exceed that of the system can provide. Vertical scaling on the other hand, allocates for resources within the same system. Unlike horizontal CPU scaling, Vertical CPU autoscalers adjust for resource limits on finer granularity e.g., millicores or fractional vCPUs.

The first type of autoscalers that came exist were based on rules and thresholds. The Kubernetes Vertical Pod Autoscaler (VPA) observes utilization over a period of time and heuristically determines resource limits based on thresholds \cite{noauthor_kubernetes_nodate}. RUBAS \cite{rattihalli_exploring_2019} uses the the sum of the median and the deviation of CPU utilization observations to recommend limits. These approaches while simple to implement often, suffer from issues of accuracy when compared to its counterparts.

Coming to machine learning and reinforcement learning based solution, Autopilot \cite{rzadca_autopilot_2020} employs moving window predictors and recommends based on Machine learning (ML). CPU usage prediction techniques have also been designed \cite{wang_predicting_2021} using Holt-Winters exponential smoothing (HW) and Long Short-Term Memory (LSTM) methods. Sinan \cite{zhang_sinan_2021} uses ML to model for SLO violations, while FIRM \cite{qiu_firm_nodate} uses reinforcement learning to not only react to SLO violations but also identify key microservices. Autothrottle \cite{wang_autothrottle_2023} uses the throttle heuristics to scale for CPUs, and uses reinforcement learning to ascertain targets based on SLOs. Barring a few, ML and RL based solutions boast improvements in accuracy, however require a long time to re-train and have a high inference overhead which leads to lower reactivity.

Shenango \cite{ousterhout_shenango_2019} and Caladan \cite{fried_caladan_2020} unlike from other autoscaling solution is a CPU scheduler that uses techniques of identifying queuing delays to achieve better performance and efficiency. However these solutions require a fundamental redesign and do not allocate for cores in the precision of granularity required by cloud providers for bandwidth sharing.

Lastly, performance aware techniques like Cilantro \cite{bhardwaj_cilantro_2023} aim to use real world metrics such as latency and throughput to build online performance models. Cilantro however, is not directly comparable as it is built to optimize utilization for a fixed cluster setting and not the elastic cloud.

All the related works discussed above make use of surrogate metrics (CPU Utilization, throttle, etc.) to model for performance behavior. CATCloud, does not rely on these proxies and rather implements light weight extensions to the OS to extract runtime information from the scheduler. This approach significantly reduces the need for models required to analyze and predict trends making CATCloud light-weight and highly reactive to sudden load changes.


\chapter{Conclusion}
\label{chp:conclusion}
CATCloud is a first in class vertical CPU autoscaler that operates directly within the OS. It minimally instruments the scheduler to extract run and yield characteristics to model the application behavior. Operating in millisecond timescales, it uses simple techniques to recommend CPU limits and offers the ability to tune both quota and period of the CPU bandwidth controller. This unique approach enables CATCloud to optimize performance by achieving high degrees of performance for the lower CPU limits, thereby reducing operating costs for cloud-based applications.

% NOTE 1: The Graduate College standards allow sections to be numbered by chapter number, section number, and subsection number. This means you can use the following commands within a chapter:
% \section{}
% \subsection{}
% \paragraph{} (does not produce a number)
% In other words, do not use the command \subsubsection{} and beyond!

% NOTE 2: The Graduate College is picky about access white space, so you should attempt to minimize access whitespace when possible. For example, Latex will move a section header and subsequent paragraph onto the next page to avoid having a section header followed by a single line of text. In this case, you should use the command "\clearpage \noindent" at the end of the first line text in the paragraph to try to bump the header and one line of text back onto the previous page.

%\chapter{Conclusions}
%\label{chp:concl}
%\input{conclusions}  % Inserts content from "conclusions.tex" here

%%%%%%%%%%%%%%%%%%%%%%%%%%%%%%%%%%%%%%%%%%%%%%%%%%%%%%%%%%%%%%%%%%%%%%%%%%%%%%%
% BIBLIOGRAPHY
%
\bibliographystyle{IEEE_ECE}
\bibliography{references}  % Put references in BibTeX format in thesisrefs.bib.

%%%%%%%%%%%%%%%%%%%%%%%%%%%%%%%%%%%%%%%%%%%%%%%%%%%%%%%%%%%%%%%%%%%%%%%%%%%%%%%
% APPENDIX
%
% NOTE: Appendices go *after* the bibliography (see here: https://grad.illinois.edu/thesis/format). However, if appendices contain citations, then you may move the appendices *before* the bibliography section.
\appendix

%\chapter{Something}
%\label{apx:something}
%\input{appendix-something}  % inserts content from "appendix-name.tex"

\backmatter

\end{document}
\endinput
